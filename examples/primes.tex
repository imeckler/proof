\newcommand{\bn}{\mathbb{N}}
\newcommand{\rs}[1]{\left|#1\right|}
\newcommand{\ps}[1]{\left(#1\right)}
\newcommand{\Set}[1]{\left\{{#1}\right\}}
\newcommand{\floor}[1]{\left\lfloor{#1}\right\rfloor}
\newcommand{\hdots}{\dots}

\begin{document}
  
  \definition{square-free}{
      \label{squarefree}
      Say an number $k \in \bn$ is \emph{square-free} if
      the only square number dividing $k$ is equal to $1$.
  }

  \definition{$\pi(n)$}{
    \label{pidef}
    For $n \in \bn$, let $\pi(n)$ be the number of primes less
    than or equal to $n$.
  }

  \definition{$\Pi(n)$}{
    \label{bigpin}
    For $n \in \bn$, let $\Pi(n)$ be the set of primes less
    than or equal to $n$.

    \comment[NB]{
      Note that $\rs{\Pi(n)} = \pi(n)$.
    }
  }

  \definition{$f : \mathcal{P}(\Pi(n)) \times \Set{1, \hdots, \floor{\sqrt{n}}} \to \bn$}{
    Define $f$ by
    $$ f\ps{\Set{p_1, \dots, p_k}, b} = p_1 \cdots p_k \cdot b $$
  }

  \lemma{Factorization lemma}{
    \label{faclem}
    \suppose{
      \item $m, n \in \bn$
      \item $m \leq n$
    } \then{
      \item
        There exists $\Set{p_1, \hdots, p_k} \in \mathcal{P}(\Pi(n))$ and
        $a \leq \sqrt{n}$ such that
        $p_1 \cdots p_k \cdot a^2 = m$.
    }

    \proof{
      \claim{For some $a, b \in \bn$ with $b$ square-free, we have $ba^2 = m$.}{
        \let{
          \item $a$ be the largest number such that $a^2$ divides $n$ 
          \item $b = \frac{n}{a^2}$
        }
        \claim{$b$ is square-free}{
          \take{\item $k \in \bn$} \suchthat{\item $k^2$ divides $b$}

          \claim{$k = 1$}{
            \cases{
              \case{$k = 0$}{\simple{This case is impossible since $0$ divides no number}}

              \case{$k = 1$}{\simple{Reflexivity.}}
              
              \case{$k > 1$}{\simple{
                If $k > 1$, then $k a > a$, contradicting the fact that
                $a$ is the largest number such that $a^2$ divides $n$.
                Thus this case is impossible.
              }}
            }
          }
          \claim{Q.E.D.}{\simple{Definition of square-free \ref{squarefree}}}
        }
        \claim{Q.E.D.}
      }
      \claim{$a \leq \sqrt{n}$}{\simple{
        $$a \leq \sqrt{m} \leq \sqrt{n}$$
        which follows from the previous claim and monotonicity of $\sqrt{-}$.
      }}
      \claim{
        There exists $\Set{p_1, \hdots, p_k} \in \mathcal{P}(\Pi(n))$ such that
        $b = p_1 \cdots p_k$
      }{
        \claim{There exist primes $p_1, \hdots, p_k$ such that $b = p_1 \cdots p_k$}{
          \simple{By the prime number theorem}
        }
        \claim{For each $i$, $p_i \in \Pi(n)$}{
          \simple{$$p_i \leq b \leq m \leq n$$ so the claim follows from the definition \ref{bigpin}.}
        }
        \claim{For any $i \neq j$, $p_i \neq p_j$}{
          \simple{
            If for some $i \neq j$, $p_i = p_j$, then $p_i^2$ divides $b$, contradicting the
            square-free-ness of $b$.
          }
        }
        \claim{Q.E.D.}
      }
    }
  }

  \theorem{There are infinitely many prime numbers}{
    \suppose{
      \item There are finitely many primes.
    }
    \then{
      \item Contradiction.
    }

    \proof{
      \suppose{\item $n \in \bn$}
      \then{\item $\pi(n) \geq \frac{1}{2} \log n$}
      \because{
        \claim{$2^{\pi(n)} \sqrt{n} \geq n$}{
          \claim{$2^{\pi(n)} \sqrt{n} \geq \rs{\mathcal{P}(\Pi(n)) \times \Set{1, \hdots, \floor{\sqrt{n}}}}$}{
            \simple{
              \begin{align*}
                \rs{\mathcal{P}(\Pi(n)) \times \Set{1, \hdots, \floor{\sqrt{n}}}}
                &= 2^{\pi(n)} \floor{\sqrt{n}}
                &\leq 2^{\pi(n)} \sqrt{n}
              \end{align*}
            }
          }
          \claim{$\rs{\mathcal{P}(\Pi(n)) \times \Set{1, \hdots, \floor{\sqrt{n}}}} \geq n$}{
            \claim{$\mrm{Im}(f) \supseteq \Set{1, \hdots, n}$}{
              \simple{
                Let $m \in \Set{1, \hdots, n}$. By \ref{faclem}, there is some
                $(P, a) \in \mathcal{P}(\Pi(n)) \times \Set{1, \hdots, \floor{\sqrt{n}}}$
                such that $f(P, a) = m$.
              }
            }
            \claim{Q.E.D.}{
              \simple{
                The last claim implies the existence of a surjection from the domain
                of $f$ (namely $\mathcal{P}(\Pi(n)) \times \Set{1, \hdots, \floor{\sqrt{n}}}$)
                to $\Set{1, \hdots, n}$, so that the cardinality of $\mrm{Dom}(f)$ is
                at least $n$.
              }
            }
          }
          \claim{Q.E.D.}{\simple{Transitivity of $\geq$.}}
        }
        \claim{Q.E.D.}{\simple{Algebra.}}
      }

      \claim{Q.E.D.}{
        \simple{
          Since the function $n \mapsto \frac{1}{2} \log n$ goes
          to infinity with $n$, so too does $\pi(n)$.
          By \ref{pidef}, this implies there are infinitely many primes.
        }
      }
    }
  }
\end{document}
