\newcommand{\bn}{\mathbb{N}}

\begin{document}
  
  \definition{square-free}{
      \label{squarefree}
      Say an number $k \in \bn$ is \emph{square-free} if
      the only square number dividing $k$ is equal to $1$.
  }

  \theorem{There are infinitely many prime numbers}{
      \suppose{
        \item There are finitely many primes.
      }
      \then{
        \item Contradiction.
      }

      \proof{
        \suppose{$n \in \bn$}
        \then{
          \item
            For some $a, b \in \bn$ with $b$ square-free, we have
            $n = a^2 b$.
        } \because{
          \let{
            \item $a$ the largest number such that $a^2$ divides $n$ 
            \item $b = \frac{n}{a^2}$.
          }

          \claim{$b$ is square-free}{
            \suppose{There is some number $k$ such that $k^2$ divides $b$}
            \then{$(k a)^2$ divides $n$}

            \claim{Q.E.D.}{
              If $k \neq 1$, then $k a > a$ contradicting the fact that
              $a$ is the largest number such that $a^2$ divides $n$.
              Thus $k = 1$ so by definition \ref{squarefree}, $b$ is
              squarefree.
            }
          }

%TODO: Counting lemma, sets surjection size lemma
%TODO: Complete set of type theoretic primitives with \allof (i.e., sigma type)
        \suppose{There are finitely many primes.}

        \then{
        }
      }
    }
  }
\end{document}
